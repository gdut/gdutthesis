\chapter{模板编写}
《广东工业大学本科生毕业设计(论文)手册》\upcite{shouce}(下文简称《手册》)中对本科毕业论文构成、格式等作了详细的规定,编写一个符合这些规定的模板是本次设计的重要内容。本章将介绍论文模板编写的具体流程。

\section{宏包}
\TeX{} Live中有多种宏包供我们选择,我们用几个常用的宏包来构建我们的论文模板。见\tabref{package}。
\begin{table}[htbp]
\begin{center}
\caption{宏包名称及对应功能}
\label{tab:package}
\begin{tabularx}{\linewidth}{Z|Z|Z|Z|Z|Z} \toprule
宏包 & 功能 & 宏包 & 功能 & 宏包 & 功能 \\\cline{1-6}
geometry & 版面设置 & fancyhdr &  版式设计 & titlesec & 标题设置\\\cline{1-6}
titletoc & 目录设置 & graphicx & 插图 & subfig & 图片子标题\\\cline{1-6} 
float & 浮动体 & array & 数组 & booktabs & 线表格\\\cline{1-6}
tabularx & 可调列宽表格 & multirow & 跨行表格 & paralist & 列表\\\cline{1-6}
amsmath amssymb bm & 公式环境 & txfonts & 字体 & ntheorem & 定理\\\cline{1-6}
xeCJK & 中文支持 & indentfirst & 首行缩进 & natbib & 引用样式\\\cline{1-6}
hyperref & 超链接 & xcolor & 表格颜色 & listings & 代码\\\cline{1-6}
fancyvrb & 抄录环境 & algorithm & 算法环境 & & \\\bottomrule
\end{tabularx}
\end{center}
\end{table}

模板使用命令\verb|\RequirePackage|加载,用法

{\centering {\verb|\RequirePackage[参数1,参数2,参数3......]{宏包}|}\\}

如:

{\centering {\verb|\RequirePackage[center,pagestyles]{titlesec}|}\\}

\section{基本设置}
\subsection{字体}
\label{sec:font}
\TeX{}的字体设置比较麻烦,因此要用到XeCJK宏包。我们定义了几种文字,如\tabref{font},以便于我们编写使用。同时也支持Adobe公司的字体库。假如运行在Linux下,默认的系统字体不适合书写文档\upcite{linuxfont},建议使用Adobe公司的字体库。
\begin{table}[htbp]
\begin{center}
\caption{字体设定}
\label{tab:font}
\begin{tabularx}{\linewidth}{ZZZZ} \toprule
字体 & 命令 & Windows 7 & Adobe\\\cline{1-4}
Times New Roman & \multicolumn{3}{c}{无需设定,\LaTeX{}自带} \\
宋体 & \verb|\song| & SimSun & Adobe Song Std \\
黑体 & \verb|\hei| & SimHei & Adobe Heiti Std \\\bottomrule
\end{tabularx}
\end{center}
\end{table}

用\verb|\setCJKfamilyfont|设置字体的自定义名称,用法

{\centering {\verb|\setCJKfamilyfont{自定名称}{字体名称}|}\\}

如:

{\centering {\verb|\setCJKfamilyfont{kai}{Adobe Kaiti Std}|}\\}

\subsection{字号}
\label{sec:fontsize}
\LaTeX{}中的字体大小一般都用pt做单位\upcite{point},跟我们平时熟悉的在Word中的四号、五号字格式不同,所以可以根据对应关系定义新指令,用来在后面的编写中方便地改变字号,这里仅提供论文要用到的字体大小,其他大小不予给出,如\tabref{fontsize}。
\begin{table}[hbpt]
\begin{center}
\caption{字号设置}
\label{tab:fontsize}
\begin{tabularx}{\linewidth}{ZZZZZZZZ}\toprule
字号 & 大小(pt)& 字号 & 大小(pt)& 字号 & 大小(pt)& 字号 & 大小(pt)\\\cline{1-8}
二号 & 22pt & 小二 & 18pt & 三号 & 16pt & 四号 & 14pt \\
小四 & 12pt & 五号 & 10.5pt & 小五 & 9pt & & \\\bottomrule
\end{tabularx}
\end{center}
\end{table}

用\verb|\newcommand{}|设置字号的自定义名称,用法

{\centering {\verb|\newcommand{命令}{\fontsize{大小}{行距}\selectfont}|}\\}

如:

{\centering {\verb|\newcommand{erhao}{\fontsize{22pt}{\baselineskip}\selectfont}|}\\}

\subsection{中文元素}
默认的页面元素的英文名,诸如Contents 为目录,Abstract 为摘要等,我们首先将他们一一中文化,需要中文化的元素如\tabref{element}
\begin{table}[htbp]
\begin{center}
\caption{元素}
\label{tab:element}
\begin{tabularx}{\linewidth}{ZZZZZZ}\toprule
元素 & 中文 & 英文 & 元素 & 中文 & 英文 \\\midrule
目录 & content & 目 录 & 致谢 & 致~谢 & Acknowledgements \\
公式 & equation & 公式 & 参考文献 & 参~考~文~献 & References \\
图片 & figure & 图 & 表格 & 表 & table\\
附录 & Appendix & 附录 & 年月 & 2013年6月 & 2013.6 \\\bottomrule
\end{tabularx}
\end{center}
\end{table}

\section{版式}
学校规定,学位论文文稿用A4 纸( 210mm×297mm )标准大小的白纸双面打印,论文装订后尺寸为标准A4 纸的尺寸,一律在左侧装订,要求装订、剪切整齐,便于使用和保存。

本模板设置上边距30mm,下边距25mm ,左边距30mm和右边距20mm ,页脚顶部到版心距离10mm,到页面底边距离15mm,由于此宏包的页脚高度不计入页边距里,所以,实际的页边距为25mm=10mm+15mm直到比对正确为止,设定如下:

{\centering {\textbackslash geometry\{top=28mm,bottom=15mm,left=30mm,right=20mm,nohead,footskip=10mm\}}\\}

\section{页脚}
《手册》中的页脚格式为绪论至附录有页脚,模板中设置一个无页脚的空页面和一个有页脚的页面。绪论之前的页面都是用无页脚模式。
\begin{lstlisting}
\fancypagestyle{plain}{\renewcommand{\headrulewidth}{0pt}\fancyhf{}} %`无页脚`
\fancypagestyle{mainpage}{
\fancyhf{}
\fancyfoot[ER,OR]{\thepage} %`页脚始终在右边`
} 
\end{lstlisting}

\section{编写格式}
当页面设置好之后,就是在论文的不同部分分别调用,一般来说论文类的书籍分为三个matter ,为前言区(前置部分),正文区(主体),后文区(附录),在《手册》中要求,绪论至附录要求有页码,且封面,摘要及目录不计入页码。

首先看前置部分,主要包括封面,摘要,目录等,实现为:
\begin{lstlisting}
\renewcommand\frontmatter{%
    \if@openright\cleardoublepage\else\clearpage\fi
    \@mainmatterfalse
	\fancyhf{}
    \pagestyle{plain}
}
\end{lstlisting}
其中\verb|\@mainmatterfalse|为停止章节计数器,即前面的章节均假如正文的章节计算。

之后为文章的正文区,采用阿拉伯数字编页码:
\begin{lstlisting}
\renewcommand\mainmatter{%
    \if@openright\cleardoublepage\else\clearpage\fi
    \@mainmattertrue
    \pagenumbering{arabic}
    \sihao
    \def\@tabular{\wuhao[1]\old@tabular} % 之后表格字体使用5号
    \pagestyle{mainpage}
  }
\end{lstlisting}
其中\verb|\normalsize|为正文的字体、字号、间距等设置,\ref{sec:normalsize}~条将会提到。

附录属于后文区,具体设置见\ref{sec:appendix}~条。

\section{摘要}
模板为摘要做了一个专门的环境,符合《手册》上的规定。
\begin{lstlisting}[language=TeX]
\newenvironment{cabstract}{%
  \titleformat*{\chapter}{\xiaosan \heiti \filcenter \bfseries}
  \chapter*{\cabstractname}
  \xiaosi
  \@afterheading} %`保留章节标题下的首行段落的空白`
{\par\vspace{1em}\par}
\end{lstlisting}
中文摘要设为不参与排序的节,所以使用了\textbackslash chapter*\{\textbackslash cabstractname\},中间上方的星号意为不参与排序。由于不需求页码,所以在Thesis.tex文件加载摘要前加入\textbackslash frontmatter。

\section{目录}
目录用\verb|\tableofcontents|即可以很方便地自动生成。按照《手册》要求,自动生成的目录还不符合格式要求,用titletoc宏包来制定目录的样式。用法:

{\centering {\verb|\titlecontents{标题层次}[左间距]{整体格式}|
\verb|{标题序号}{标题 内容}{指引线和页码}[下间距]|}\\}

如:

{\centering {\verb|\titlecontents{chapter}[0pt]{\vspace{6bp} \heiti \xiaosi[1]}|
\verb|{\thecontentslabel\quad}{}{\titlerule*{.}\contentspage}|}\\}

\section{正文部分}
\subsection{正文标题}
《手册》规定了正文各级标题的格式,可调用titlesec宏包来设置。用法,

{\centering{\verb|\titleformat{标题层次}{整体格式}{标题 序号}{序号与内容间距}{标题内容}|}\\}

如:

{\centering {\verb|\titleformat{\chapter}{\filcenter \heiti \bfseries \sanhao}{\thechapter}{1em}{}|}\\}

目录的层次深度为条,即第三级,所以模板设置为\textbackslash setcounter\{secnumdepth\}\{3\}。
\subsection{正文字体}
\label{sec:normalsize}
首先确定正文中使用的字体,文档要求正文字体为小四,行距为1.5倍,中文字体为宋体,英文为Times New Roman。

由于大部分人习惯用Word里的行距,所以经过微调后,设置\textbackslash baselinestretch为1.25倍。

\subsection{正文段落}
文档设定为首行缩进2个字符,这一个命令需要在文档开始时自动执行:

{\centering {\verb|\setlength{\parindent}{2.5em}|}\\}
2.5em值得是2.5个字符量,由2em经过实际微调而得到。

定义段落间距,段前间距以及段后间距都为0。

{\centering {\verb|\setlength{\parskip}{0bp \@plus .5bp \@minus .5bp}|}\\}

\section{浮动对象}
浮动对象针对的目标是图片表格,标题为五号字体,图片标题在下,表格标题在上。具体设置:
\begin{lstlisting}[language=TeX]
\setlength{\floatsep}{12bp \@plus 2bp \@minus 1bp}
\setlength{\intextsep}{12bp \@plus 2bp \@minus 1bp}
\setlength{\textfloatsep}{12bp \@plus 2bp \@minus 1bp}
\setlength{\@fptop}{0bp \@plus1.0fil}
\setlength{\@fpsep}{8bp \@plus2.0fil}
\setlength{\@fpbot}{0bp \@plus1.0fil}
\renewcommand{\textfraction}{0.15}
\renewcommand{\topfraction}{0.85}
\renewcommand{\bottomfraction}{0.65}
\renewcommand{\floatpagefraction}{0.80}
\let\old@tabular\@tabular
\DeclareCaptionLabelFormat{tabularlabel}{{\wuhao \heiti #1~\rmfamily #2}}
\DeclareCaptionLabelFormat{figurelabel}{{\wuhao \song #1~\rmfamily #2}}
\DeclareCaptionLabelSeparator{floatsong}{\hspace{1em}}
\DeclareCaptionFont{tabularcap}{\wuhao \bfseries \heiti }
\DeclareCaptionFont{figurecap}{\wuhao }
\captionsetup[table]{position=top,belowskip=-0.2em,aboveskip=0.1em,labelformat=tabularlabel,labelsep=floatsong,font=tabularcap}
\captionsetup[figure]{position=bottom,belowskip=-0.2em,aboveskip=0.1em,labelformat=figurelabel,labelsep=floatsong,font=figurecap}
\captionsetup[subfloat]{justification=centering}
\renewcommand{\thesubfigure}{(\alph{subfigure})}
\renewcommand{\thesubtable}{(\alph{subtable})}
\end{lstlisting}

\section{封面}
\label{sec:cover}
封面的组成是两个不浮动的图片(校徽、校名称),加上\textbackslash vspace\{\}间距控制和几行文字组成。其中的图片大小皆符合《手册》给出的大小。而间距控制根据《手册》以及实际生成的封面进行对比微调。直接给出代码:

\begin{lstlisting}[language=TeX]
\begin{flushleft}
\hspace{8.5mm}\includegraphics[height=2.19cm,width=2.21cm]{xiaohui.jpg} %`校徽`
\end{flushleft}
\begin{center} 
\includegraphics[height=2.96cm,width=10.56cm]{mingchen.jpg} %`学校名字`
\end{center}
\vspace{6.5mm}
\begin{center}
{\heiti \yihao {`本科毕业设计(论文)`}}
\end{center}
\vspace{20mm}
\begin{center}
{\heiti \erhao \textbf{`XXXXXXXXXXXXXXXXXXXXXX`}}
\end{center}
\vspace{25mm}
\begin{center}
{\sanhao \heiti {`学`\hspace{2em}`院`\hspace{3mm}\dlmu{\textbf{`物理光电工程学院`}}}}\\\vspace{2.5mm}
{\sanhao \heiti {`专`\hspace{2em}`业`\hspace{3mm}\dlmu{\textbf{`电子科学与技术`}}}}\\\vspace{2.5mm}
{\sanhao \heiti {`年级班别`\hspace{3mm}\dlmu{\textbf{`XXXXXX`}}}}\\\vspace{2.5mm}
{\sanhao \heiti {`学`\hspace{2em}`号`\hspace{3mm}\dlmu{\textbf{XXXXXXXX}}}}\\\vspace{2.5mm}
{\sanhao \heiti {`学生姓名`\hspace{3mm}\dlmu{\textbf{`XX`}}}}\\\vspace{2.5mm}
{\sanhao \heiti {`指导教师`\hspace{3mm}\dlmu{\textbf{`XX`}}}}
\end{center}
\vspace{20mm}
\begin{center}
\makeatletter
\sanhao \heiti {\CJK@today}
\makeatother
\end{center}
\end{lstlisting}

假如需要修改封面,直接在Cover.tex修改即可。

\section{参考文献}
根据《手册》上的规定,模板为摘要做了一个专门的环境。代码如下:

\begin{lstlisting}[language=TeX]
\renewenvironment{thebibliography}[1]{%
  \thispagestyle{emptypage}
  \chapter*{\bibname}%
  \addcontentsline{toc}{chapter}{`参考文献`}
  \list{\@biblabel{\@arabic\c@enumiv}}%
  {\renewcommand{\makelabel}[1]{##1\hfill}
    \setlength{\labelsep}{1ex}
    \setlength{\itemindent}{0pt}
    \setlength{\leftmargin}{\labelwidth+\labelsep}
    \addtolength{\itemsep}{-1em}
    \usecounter{enumiv}%
    \let\p@enumiv\@empty
    \renewcommand\theenumiv{\@arabic\c@enumiv}}%
  \sloppy\frenchspacing
  \clubpenalty4000%
  \@clubpenalty \clubpenalty
  \widowpenalty4000%
  \interlinepenalty4000%
  \sfcode‘\.\@m
}
\end{lstlisting}

根据《手册》中的具体打印实例,在目录中,“参考文献”各个字中间无须加入空格。但在章节标题时需要加入空格。所以使用\textbackslash addcontentsline\{toc\}\{chapter\}\{参考文献\},以区分章节标题及目录标题。

\section{致谢}
致谢的要求如“参考文献”。在目录中,各个字中间无须加入空格,但在章节标题时需要加入空格。所以,另外为致谢做了一个环境。代码如下:

\begin{lstlisting}[language=TeX]
\newenvironment{ack}{%
  \thispagestyle{emptypage}
  \chapter*{\ackname}%
  \addcontentsline{toc}{chapter}{`致谢`}%
  \xiaos
  \@mainmatterture
  \@afterheading
}
{\par\vspace{2em}\par}
\end{lstlisting}

\section{附录}
\label{sec:appendix}
最后是附录部分,由于他的章节标题与正文中不一样(不是第几章,而是附录A),同时《手册》中的实际打印效果是“目录中的附录条目不带附录标题内容”同时“附录实际页面中标题既带有附录标签同时带有附录标题内容”。这个设定与\LaTeX{}自动生成的有比较大的出入。所以从新设定了三个计数器以及一个附录环境。

代码如下:

\begin{lstlisting}[language=TeX]
\newcounter{appx}[chapter]
\renewcommand{\theappx}{\Alph{appx}}
\newcommand{\appx}[1]{\refstepcounter{appx}%
\begin{center}
\heiti \bfseries \sanhao{\par{`附录`}\textbf{\theappx}\quad#1}
\end{center}
\vspace{0.8em}}

\newcounter{appxtab}[appx]
\renewcommand{\theappxtab}{\theappx\arabic{appxtab}}
\newcommand{\appxtab}{\refstepcounter{appxtab}\centering \heiti \bfseries {\par{`表`\hspace{0.2em}}\theappxtab}\quad}

\newcounter{appxfig}[appx]
\renewcommand{\theappxfig}{\theappx\arabic{appxfig}}
\newcommand{\appxfig}{\refstepcounter{appxfig}\centering \song{\par{`图`\hspace{0.2em}}\theappxfig\quad}}

\newcommand{\appxtoc}[1]{\appx{#1}\par\addcontentsline{toc}{chapter}{`附录`\theappx}}

\newenvironment{appxchp}{%
\@afterheading%
\clearpage}{}
\end{lstlisting}

其中包含了附录章节、附录表格、附录图片计数器。\textbackslash appxtoc被定义为附录章的标题,同时为目录添加一条附录目录。《手册》中不要求附录中有节条等多级标题,所以没加入此功能。
